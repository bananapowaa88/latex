\input amstex
\input amssym.def
\UseAMSsymbols

\magnification=1000
\hsize=12.5cm
\font\bigletter=cmr10 scaled \magstep1
\font\bigbold=cmbx10 scaled \magstep1
\font\smallcap=cmcsc8
\font\bigcap=cmcsc10
\font\headl=cmr8
\font\cursive=cmti10
\nopagenumbers

\headline{\smallcap\centerline{1.4. the skolem-mahler-lech theorem, multiplicity and growth \hfil 45}}
\vskip\baselineskip
{\bigcap Theorem} 1.27. {\cursive Let $\Bbb{K}$ be a field of positive characteristic, and let G be a finitely-generated subgroup of $\Bbb{K^*}$. Suppose that there are constants $a_1,\ldots ,a_n$ in $\Bbb{K}$ for which the equation}
$$a_1x_1+\cdots +a_nx_n=1$$
{\cursive has a broad set of solutions $(x_1,\ldots ,x_n\in$ G. Then there exist $b_1,\ldots ,b_n \in \Bbb{K}$ and $g_1,\ldots ,g_n \in$ G with $g_i\neq 1$ and $g_i/g_j \neq 1$ for $i\neq j$ (if $n\geq 2$) such that the equation}
$$b_1g_1^k+\cdots +b_ng_n^k=1$$
{\cursive has infinitely many solutions $k \in \Bbb{N}$.}
\vskip\baselineskip
Multivariate equations of the form
$$\sum_{i=1}^{m}A_i(x_1,\ldots ,x_r)\prod_{\nu=1}^{r}\alpha_{i\nu}^{x_\nu}=0$$
are considered in [{\bf 1011}], where the coefficients $A_i$ are polynomials and the `char\-acteristic roots' $\alpha_{i\nu}$ are elements of an algebraic number field $\Bbb{K}$ of degree $d$ over $\Bbb{Q}$. In this many variable case a uniform upper bound for the number of {\sl all} solutions $(x_1,\ldots ,x_r)$ cannot be expected, so a suitable equivalence relation on the solutions must be factored out and then the number of non-equivalent solutions can be esti\-mated in terms of $m, r, d,$ the number $\omega$ of prime ideal divisors of the $\alpha_{i\nu}$, and the largest degree of the polynomials $A_i$. In [{\bf 1013}] these bounds are improved, and the dependence on $\omega$ is eliminated.

Another natural generalization of linear recurrence sequences are the se\-quences satisfying the recurrence equation (1.5) with rational functions in $x$ as coefficients. Just as the class of linear recurrence sequences coincides with the class of sequences of Taylor coefficients of rational functions, the class of these general\-ized sequences coincides with the class of sequences of power series coefficients of functions satis\-fying linear differential equations with polynomial coefficients.

The first analogue of the Skolem--Mahler--Lech Theorem for such sequences is give in [{\bf 91}] and then, for these and several further types of sequences, in [{\bf 595}]. Some arguments in [{\bf 595}] turned out to be faulty and were fixed in [{\bf 97}]. These papers on power series coefficients of solutions of various differential equations in particular deal with linear combinations of the form
$$F(X)=\sum_{i=1}^{m}A_i(X)J_s(\beta_iX^t)$$
where the $A_i$ are formal power series over $\overline{\Bbb{Q}}$ (satisfying some mild conditions) and
$$J_s(X)=\sum_{j=1}^{\infty}{X^j \over{(j!)^s}}.$$
This class is quite wide and includes many interesting functions (see Corollaries 1-3 of the main theorem of [{\bf 97}]).

For $q$-recurrence sequences satisfying (1.6), an analog of the Skolem--Mahler--Lech Theorem is given in [{\bf 92}].

All these results are qualitative; van der Poorten and Shparlinski provide the first quantitative results in this direction in [{\bf 936}]. These results were substantially improved by Bezivin [{\bf 95}], but they remain much weaker then those for the case of linear recurrence sequences.
\bye 
